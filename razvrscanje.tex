\chapter{Razvrščanje}

\section{Osnovna razvrščanja}

\intro{Osnovni razvrščevalni algoritmi so:
\begin{itemize}
	\item prvi pride, prvi melje (FCFS -- first come, first serve),
	\item najkrajši posel najprej (SJF -- shortest job first),
	\item prevzemni najkrajši posel najprej (PSJF -- preemptive shortest job first) in
	\item krožno razvrščanje (RR -- round robin).
\end{itemize}
}


\begin{Exercise}
Za razvrščanje procesov v spodnji tabeli uporabi algoritma a) FCFS in b) SJF:
\par\vspace{5pt}
{\centering
\begin{tabular}{r|ccccc}
	proces & A & B & C & D & E \\
	\hline
	trajanje & 10 & 20 & 30 & 40 & 50 \\
	čas prihoda & 0 & 5 & 10 & 15 & 20 \\
\end{tabular}\\}
\end{Exercise}
\ans{
Za oba FCFS in SJF je enaka razporeditev procesov. V diagramu je znotraj okvira zapisan proces in čas njegovega izvajanja, nad diagramom so procesi, kot so prihajali v sistem, pod diagramom pa je čas.
\vspace{-2em}
\begin{center}
\includegraphics[width=.9\textwidth]{razvrscanje/1.1-FCFS,SJF.pdf}\\
\begin{tabular}{c|cc|cc|cc}
proces & trajanje & prihod & začetek & odhod & odzivni čas & čas obdelave \\
\hline
A & 10 &  0 &   0 & 10 & 0 & 10 \\
B & 20 &  5 & 10 & 30 & 5 & 25 \\
C & 30 & 10 & 30 & 60 & 20 & 50 \\
D & 40 & 15 & 60 & 100 & 45 & 85 \\
E & 50 & 20 & 100 & 150 & 80 & 130 \\
\hline
& & & & & 30 & 60
\end{tabular}
\end{center}
}


\begin{Exercise}
Za razvrščanje procesov v spodnji tabeli uporabi algoritma a) FCFS in b) SJF.
\par\vspace{5pt}
{\centering
\begin{tabular}{r|ccccc}
	proces & A & B & C & D & E \\
	\hline
	trajanje & 50 & 40 & 30 & 20 & 10 \\
	čas prihoda & 0 & 5 & 10 & 15 & 20 \\
\end{tabular}\\}
\end{Exercise}
\ans{
a) FCFS\\
\begin{center}
\includegraphics[width=.9\textwidth]{razvrscanje/1.2-FCFS.pdf}\\
\begin{tabular}{c|cc|cc|cc}
proces & trajanje & prihod & začetek & odhod & odzivni čas & čas obdelave \\
\hline
A & 50 &  0 &   0 & 50 & 0 & 50 \\
B & 40 &  5 & 50 & 90 & 45 & 85 \\
C & 30 & 10 & 90 & 120 & 80 & 110 \\
D & 20 & 15 & 120 & 140 & 105 & 125 \\
E & 10 & 20 & 140 & 150 & 120 & 130 \\
\hline
& & & & & 70 & 100
\end{tabular}
\end{center}
b) SJF:
\begin{center}
\includegraphics[width=.9\textwidth]{razvrscanje/1.2-SJF.pdf}\\
\begin{tabular}{c|cc|cc|cc}
proces & trajanje & prihod & začetek & odhod & odzivni čas & čas obdelave \\
\hline
A & 50 &  0 &   0 & 50 & 0 & 50 \\
B & 40 &  5 & 110 & 150 & 105 & 145 \\
C & 30 & 10 & 80 & 110 & 70 & 100 \\
D & 20 & 15 & 60 & 80 & 45 & 65 \\
E & 10 & 20 & 50 & 60 & 30 & 40 \\
\hline
& & & & & 50 & 80
\end{tabular}
\end{center}
}


\begin{Exercise}
Za razvrščanje procesov v spodnji tabeli uporabi algoritma a) FCFS in b) SJF.
\par\vspace{5pt}
{\centering
\begin{tabular}{r|ccccc}
	proces & A & B & C & D & E \\
	\hline
	trajanje & 10 & 30 & 30 & 10 & 20 \\
	čas prihoda & 0 & 0 & 10 & 10 & 20 \\
\end{tabular}\\}
\end{Exercise}
\ans{
a) FCFS:
\begin{center}
\includegraphics[width=.9\textwidth]{razvrscanje/1.3-FCFS.pdf}\\
\begin{tabular}{c|cc|cc|cc}
proces & trajanje & prihod & začetek & odhod & odzivni čas & čas obdelave \\
\hline
A & 10 &  0 &   0 & 10 & 0 & 10 \\
B & 30 &  0 & 10 & 40 & 10 & 40 \\
C & 30 & 10 & 40 & 70 & 30 & 60 \\
D & 10 & 10 & 70 & 80 & 60 & 70 \\
E & 20 & 20 & 80 & 100 & 60 & 80 \\
\hline
& & & & & 32 & 52
\end{tabular}
\end{center}
b) SJF:
\begin{center}
\includegraphics[width=.9\textwidth]{razvrscanje/1.3-SJF.pdf}\\
\begin{tabular}{c|cc|cc|cc}
proces & trajanje & prihod & začetek & odhod & odzivni čas & čas obdelave \\
\hline
A & 10 &  0 &   0 & 10 & 0 & 10 \\
B & 30 &  0 & 40 & 70 & 40 & 70  \\
C & 30 & 10 & 70 & 100 & 60 & 90 \\
D & 10 & 10 & 10 & 20 & 0 & 10 \\
E & 20 & 20 & 20 & 40 & 0 & 20 \\
\hline
& & & & & 20 & 40
\end{tabular}
\end{center}
}


\begin{Exercise}
Za razvrščanje procesov v spodnji tabeli uporabi algoritma a) SJF in b) PSJF.
\par\vspace{5pt}
{\centering
	\begin{tabular}{r|ccccc}
		proces & A & B & C & D & E \\
		\hline
		trajanje & 25 & 10 & 15 & 10 & 5 \\
		čas prihoda & 0 & 5 & 10 & 15 & 30 \\
	\end{tabular}\\}
\end{Exercise}
\ans{
a) SJF:
\begin{center}
\includegraphics[width=.8\textwidth]{razvrscanje/1.4-SJF.pdf}\\
\begin{tabular}{c|cc|cc|cc}
proces & trajanje & prihod & začetek & odhod & odzivni čas & čas obdelave \\
\hline
A & 25 &  0 &   0 & 25 & 0 & 25 \\
B & 10 &  5 & 25 & 35 & 20 & 30 \\
C & 15 & 10 & 50 & 65 & 40 & 55 \\
D & 10 & 15 & 40 & 50 & 25 & 35 \\
E &  5 & 30 & 35 & 40 & 5 & 10 \\
\hline
& & & & & 18 & 31
\end{tabular}
\end{center}
b) PSJF:
\begin{center}
\includegraphics[width=.8\textwidth]{razvrscanje/1.4-PSJF.pdf}\\
\begin{tabular}{c|cc|cc|cc}
proces & trajanje & prihod & začetek & odhod & odzivni čas & čas obdelave \\
\hline
A & 25 &  0 &   0 & 65 & 0 & 65 \\
B & 10 &  5 &   5 & 15 &  0 & 10 \\
C & 15 & 10 & 25 & 45 & 15 & 35 \\
D & 10 & 15 & 15 & 25 & 0 & 10 \\
E &  5 & 30 & 30 & 35 & 0 & 5 \\
\hline
& & & & & 3 & 25
\end{tabular}
\end{center}
}


\begin{Exercise}
Za razvrščanje procesov v spodnji tabeli uporabi algoritem RR. Pri tem uporabi časovno rezino: a) 10 časovnih enot in b) 5 časovnih enot.
\par\vspace{5pt}
{\centering
\begin{tabular}{r|ccc}
	proces & A & B & C \\
	\hline
	trajanje & 15 & 10 & 5  \\
	čas prihoda & 0 & 5 & 5 \\
\end{tabular}\\}
\end{Exercise}
\ans{
Narišemo le diagrama za obe časovni rezini.
\begin{center}
\includegraphics[width=.6\textwidth]{razvrscanje/1.5-RR.pdf}
\end{center}
}


\section{Efekt konvoja}

\intro{Naslednji dve nalogi sta namenjeni razlagi efekta konvoja na čas obdelave.}

\begin{Exercise}
Za razvrščanje procesov v dani tabeli uporabi optimalno razvrstitev glede na čas obdelave.
\par\vspace{5pt}
{\centering
\begin{tabular}{r|cccc}
	proces & A & B & C & D \\
	\hline
	trajanje & 7 & 2 & 19 & 6  \\
	čas prihoda & 0 & 0 & 0 & 0 \\
\end{tabular}\\}
\end{Exercise}
\ans{
OPT je SJF.
\begin{center}
\includegraphics[width=.9\textwidth]{razvrscanje/1.6-OPT.pdf}\\
\begin{tabular}{c|cc|cc|cc}
proces & trajanje & prihod & začetek & odhod & odzivni čas & čas obdelave \\
\hline
A &   7 &  0 & 8 & 15 & 8 & 15 \\
B &   2 &  0 & 0 & 2 & 0 & 2 \\
C & 19 & 0 & 15 & 34 & 15 & 34 \\
D &   6 & 0 & 2 & 8 & 2 & 8 \\
\hline
& & & & & 6,25 & 14,75
\end{tabular}
\end{center}
}


\begin{Exercise}
Za razvrščanje procesov v dani tabeli uporabi algoritme FCFS, SJF in PSJF. V primeru dvoumnosti glede časa prihoda, procese razvrstimo leksikografsko.
\par\vspace{5pt}
{\centering
\begin{tabular}{r|cccc}
	proces & A & B & C & D \\
	\hline
	trajanje & 7 & 2 & 19 & 6  \\
	čas prihoda & 1 & 1 & 0 & 1 \\
\end{tabular}\\}
\end{Exercise}
\ans{
FCFS:
\begin{center}
\includegraphics[width=.9\textwidth]{razvrscanje/1.7-FCFS.pdf}\\
\begin{tabular}{c|cc|cc|cc}
proces & trajanje & prihod & začetek & odhod & odzivni čas & čas obdelave \\
\hline
A &   7 &  1 & 19 & 26 & 18 & 25 \\
B &   2 &  1 & 26 & 28 & 25 & 27 \\
C & 19 & 0 &  0 & 19 & 0 & 19 \\
D &   6 & 1 & 28 & 34 & 27 & 33 \\
\hline
& & & & & 17,5 & 26
\end{tabular}
\end{center}
SJF:
\begin{center}
\includegraphics[width=.9\textwidth]{razvrscanje/1.7-SJF.pdf}\\
\begin{tabular}{c|cc|cc|cc}
proces & trajanje & prihod & začetek & odhod & odzivni čas & čas obdelave \\
\hline
A &   7 &  1 & 27 & 34 & 26 & 33 \\
B &   2 &  1 & 19 & 21 & 18 & 20 \\
C & 19 & 0 &  0 & 19 & 0 & 19 \\
D &   6 & 1 & 21 & 27 & 20 & 26 \\
\hline
& & & & & 16 & 24,5
\end{tabular}
\end{center}
PSJF:
\begin{center}
\includegraphics[width=.9\textwidth]{razvrscanje/1.7-PSJF.pdf}\\
\begin{tabular}{c|cc|cc|cc}
proces & trajanje & prihod & začetek & odhod & odzivni čas & čas obdelave \\
\hline
A &   7 &  1 & 9 & 16 & 8 & 15 \\
B &   2 &  1 & 1  & 3 & 0 & 2 \\
C & 19 & 0 & 0 & 34 & 0 & 34 \\
D &   6 & 1 & 3 & 9 & 2 & 8 \\
\hline
& & & & & 6,5 & 14,75
\end{tabular}
\end{center}
}

\section{Prednostna razvrščanja}


\begin{Exercise}
Za razvrščanje procesov v spodnji tabeli uporabi algoritem HPF, pri čemer
\begin{enumerate}
	\item uporabi sodelovano razvrščanje,
	\item uporabi prevzemno razvrščanje. 
\end{enumerate} 
\par\vspace{5pt}
{\centering
\begin{tabular}{r|ccc}
	proces & A & B & C \\
	\hline
	trajanje & 20 & 20 & 20  \\
	čas prihoda & 0 & 10 & 20 \\
	prioriteta & 1 & 2 & 3 \\
\end{tabular}\\}
\end{Exercise}
\begin{Answer}
Sodelovalno HPF razvrščanje:
\begin{center}
\includegraphics[width=.9\textwidth]{razvrscanje/1.8-HPF-cooperative.pdf}\\
\begin{tabular}{c|ccc|cc|cc}
proces & trajanje & prihod & prioriteta & začetek & odhod & odzivni čas & čas obdelave \\
\hline
A & 20 &  0 &  1 &   0 & 20 & 0 & 20 \\
B & 20 & 10 & 2 & 40 & 60 & 30 & 50 \\
C & 20 & 20 & 3 & 20 & 40 & 0 & 20 \\
\hline
& & & & & & 10 & 30
\end{tabular}
\end{center}
Prevzemno HPF razvrščanje:
\begin{center}
\includegraphics[width=.9\textwidth]{razvrscanje/1.8-HPF-preemptive.pdf}\\
\begin{tabular}{c|ccc|cc|cc}
proces & trajanje & prihod & prioriteta & začetek & odhod & odzivni čas & čas obdelave \\
\hline
A & 20 &  0 &  1 &   0 & 60 & 0 & 60 \\
B & 20 & 10 & 2 & 10 & 50 & 0 & 40 \\
C & 20 & 20 & 3 & 20 & 40 & 0 & 20 \\
\hline
& & & & & & 0 & 40
\end{tabular}
\end{center}
\end{Answer}


\exe{Procesi A, B in C imajo 10, 20 in 40 prepustnic zaporedoma. Razvrščevalni algoritem je izbral prepustnice št. 15, 5, 10, 42, 9, 66. Zapiši vrstni red, ki ustreza izbranimi prepustnicam. Prepustnice številčimo od 0 naprej.}
\ans{Vrstni red: B,A,B,C,A,C
\begin{center}
\includegraphics[width=.9\textwidth]{razvrscanje/1.9-tickets.pdf}
\end{center}
}


\exe{Procesi A, B in C imajo dolžine korakov 10, 15 in 30 zaporedoma. Uporabi koračno razvrščanje in zapiši prvih 10 izbranih procesov, pri čemer predpostavljaj \vic{neskončno} trajanje procesov.}
\ans{A,B,C,A,B,A,A,B,C, nato se vse skupaj ponovi neskončnokrat. Časovna rezina je lahko poljubna.
\begin{center}
\includegraphics[width=.4\textwidth]{razvrscanje/1.10-stride.pdf}
\end{center}
}


\exe{Procesi A, B in C imajo dolžine korakov 10, 15 in 30 zaporedoma, njihovo trajanje pa je 40, 20, 30, zaporedoma. Uporabi koračno razvrščanje s časovno rezino 10 in zapiši prvih 10 izbranih procesov.}
\ans{A,B,C,A,B,A,A,C,C
\begin{center}
\includegraphics[width=.5\textwidth]{razvrscanje/1.11-stride.pdf}
\end{center}
}


\section{Ocenjevanje trajanja procesov}

\intro{Trajanje procesa ocenjujemo po formuli
$$ T_i = \alpha\cdot t_i + (1-\alpha)\cdot T_{i-1}, $$
kjer je $\alpha$ faktor pozabljanja, $t_i$ zadnje trajanje procesa in $T_i$ zadnja ocena trajanja. Nova ocena trajanja procesa pa je $T_{i+1}$. Začetno oceno označimo s $T_0$.}


\begin{Exercise}
Naj bo $\alpha=0.5$ faktor pozabljanja in $T_0=5$ začetna ocena trajanja procesa. Dopolni tabelo z ocenami trajanja procesov:
\par
{\centering
	\begin{tabular}{c|cccc}
		$i$    & 1 & 2 & 3 & 4 \\ 
		\hline
		$t_i$ & 11 & 4 & 2 & 20 \\
		$T_i$ &  &  &  &  \\
	\end{tabular}\\
}
\end{Exercise}
\ans{
	\begin{tabular}{c|cccc}
	$i$    & 1 & 2 & 3 & 4 \\ 
	\hline
	$t_i$ & 11 & 4 & 2 & 20 \\
	$T_i$ & 8 & 6 & 4 & 12 \\
	\end{tabular}\\
}


\begin{Exercise}
Naj bo $\alpha=0.2$ faktor pozabljanja in $T_0=10$ začetna ocena trajanja procesa. Dopolni tabelo z ocenami trajanja procesov:
\par
{\centering
	\begin{tabular}{c|cccc}
	$i$    & 1 & 2 & 3 & 4 \\ 
	\hline
	$t_i$ & 10  & 15 & 16 & 22\\
	$T_i$ & & & & \\
	\end{tabular}\\
}
\end{Exercise}
\ans{
\begin{tabular}{c|cccc}
	$i$    & 1 & 2 & 3 & 4 \\ 
	\hline
	$t_i$ & 10  & 15 & 16 & 22\\
	$T_i$ & 10 & 11 & 12 & 14 \\
\end{tabular}\\
}


\section{Namigi in rešitve izbranih nalog}

\shipoutAnswer
